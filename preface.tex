\documentclass[12pt,letterpaper]{article}

% encoding and fonts first
\usepackage[utf8]{inputenc}
\usepackage[T1]{fontenc}
\usepackage{microtype}
\usepackage[tt=false, type1=true]{libertine}
\sloppy

\usepackage{geometry}
%% For preface:
\geometry{textwidth=14cm,textheight=20cm}
%% For committee listings and sponsor pages:
%\geometry{textwidth=18cm,textheight=23.5cm}

%% For format `acmsmall'
%\geometry{twoside=true,
%          includeheadfoot, head=13pt, foot=2pc,
%          paperwidth=6.75in, paperheight=10in,
%          top=58pt, bottom=44pt, inner=46pt, outer=46pt,
%          marginparwidth=2pc,heightrounded
%         }

%% For IEEE conferences and workshops, please uncomment the following line to use Times.
%\usepackage{times}


%-------------------------------------------------------------------------
\begin{document}

\title{\sffamily\bfseries Welcome from the Chairs}
\date{}

\maketitle
\thispagestyle{empty}
\pagestyle{empty}

Welcome to the 12th ACM Workshop on Virtual Machines and Language Implementations (VMIL’20). The workshop is co-located with SPLASH’20 in Chicago, US.

The workshop aims at advancing the state of the art on the design and implementation of programming systems, with virtual machines---broadly defined---as a focus.
Due to COVID-19, many computing conferences, including VMIL, have struggled to obtain submissions. The workshop had two submission deadlines. The first
deadline was intended for mature work to be published as part of these proceedings.
The second deadline was intended for work-in-progress papers, new ideas, and
emerging results. In total, the workshop attracted one full-length paper and no
work-in-progress submissions. Due to the low number of submissions, the submitted paper received eight reviews. The program committee accepted this submission
for publication and presentation in the workshop.
Additionally, this year we have four invited talks from industry and academia.
Matthew Flatt on Racket’s intermediate language for control. Mark Stoodley on
JitBuilder2.0, a new framework for optimizing DLS compilers. Chris Seaton on new
tools to help understanding the Graal IR, and fnally, Vyacheslav Egorov on the
evolution of the Dart programming language.
We would like to thank the authors for their submission, and all program
committee members for their high-quality and insightful reviews. Finally, we thank
all attendees for their participation.

\bigskip
\noindent
       \hfill Gregor Richards, Manuel Rigger\\
September 2021 \hfill VMIL Program Co-Chairs

\end{document}
