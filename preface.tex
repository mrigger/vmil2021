\documentclass[12pt,letterpaper]{article}

% encoding and fonts first
\usepackage[utf8]{inputenc}
\usepackage[T1]{fontenc}
\usepackage{microtype}
\usepackage[tt=false, type1=true]{libertine}
\sloppy

\usepackage{xspace}
\usepackage{geometry}
%% For preface:
\geometry{textwidth=14cm,textheight=20cm}
%% For committee listings and sponsor pages:
%\geometry{textwidth=18cm,textheight=23.5cm}

%% For format `acmsmall'
%\geometry{twoside=true,
%          includeheadfoot, head=13pt, foot=2pc,
%          paperwidth=6.75in, paperheight=10in,
%          top=58pt, bottom=44pt, inner=46pt, outer=46pt,
%          marginparwidth=2pc,heightrounded
%         }

%% For IEEE conferences and workshops, please uncomment the following line to use Times.
%\usepackage{times}


\newcommand{\vmil}{VMIL'21\xspace}


%-------------------------------------------------------------------------
\begin{document}

\title{\sffamily\bfseries Welcome from the Chairs}
\date{}

\maketitle
\thispagestyle{empty}
\pagestyle{empty}

Welcome to the \emph{13th ACM SIGPLAN International Workshop on Virtual Machines and Intermediate Languages (VMIL 2021)}. The workshop is co-located with SPLASH'21 and will be held as a hybrid event in Chicago, US.
The workshop aims at advancing the state of the art on the design and implementation of programming systems, with virtual machines---broadly defined---as a focus.
VMIL could attract a sizable number of seven submissions this year, after a submission low due to COVID-19 last year.

The workshop had two submission deadlines. The first deadline was intended for mature work to be published as part of these proceedings.
The second deadline was intended for work-in-progress papers, new ideas, and emerging results.
In total, the workshop attracted two full-length papers, four work-in-progress papers, and one position paper.
Each paper received three reviews.
The program committee accepted the two full-length papers and three work-in-progress papers for presentation in the workshop.
Additionally, three of these papers were submitted for publication and archival in the ACM DL.
Additionally, this year, we have invited three speakers from industry and academia to give keynotes.

We would like to thank the authors for their submission, and all program
committee members for their high-quality and insightful reviews. Finally, we thank
all attendees for their participation.

\bigskip
\noindent
       \hfill Gregor Richards, Manuel Rigger\\
September 2021 \hfill VMIL Program Co-Chairs


\clearpage{}

\section*{Work-in-Progress Presentations}

In addition to the papers included in these proceedings,
\vmil{} featured the following talks:
~\\

\noindent
\emph{Lightweight IOT abstractions for Embedded WebAssembly}\\
Tom Lauwaerts, Robbert Gurdeep Singh, Christophe Scholliers
~\\

\noindent
\emph{WOOD: Extending a WebAssembly VM with Out-of-Place Debugging for IoT applications}\\
Carlos Javier Rojas Castillo, Matteo Marra, Jim Bauwens, Elisa Gonzalez Boix
~\\

\clearpage


\section*{\vmil Workshop Organization}

\begin{table*}[h!]
\begin{tabular}{p{2.2cm}p{10.9cm}}
\textbf{Steering\newline{}Committee} &
Daniele Bonetta, Oracle Labs, USA\newline
Marc Feeley, Université de Montréal, Canada\newline
Juan Fumero, The University of Manchester, UK\newline
Tony Hosking, Australian National University, Australia\newline
Yu David Liu, State University of New York (SUNY), USA
Stephen Kell, King's College London, UK\newline
Witawas Srisa-an, University of Nebraska-Lincoln, USA\newline
\newline
\\
\textbf{Program\newline Committee} &
Daniele Bonetta, Oracle Labs, USA\newline
Marc Feeley, Université de Montréal, Canada\newline
Matthew Flatt, University of Utah, USA\newline
Olivier Flückiger, Northeastern University, USA\newline
Elisa Gonzalez Boix, Vrije Universiteit Brussel, Belgium\newline
Tianxiao Gu, Alibaba Group, USA\newline
Fernando M. Q. Pereira, Federal University of Minas Gerais, Brazil\newline
Guannan Wei, Purdue University, USA\newline
\\
\textbf{Program Chairs} &
Gregor Richards, University of Waterloo, Canada\newline
Manuel Rigger, ETH Zurich, Switzerland\newline

\end{tabular}
\end{table*}


\end{document}
